% File name: /home/adh/Projects/LaTeXClasses/adharticle_test.tex Document to
% test functionality of adharticle class.  Author: adh Date: Wed 24 Dec 2014
% 17:23

\documentclass[electronic,timesopt]{adharticle} % Standard document class

% Define title here:
\title{A customised \LaTeX{} article class} \author{Andy Holt} \date{\today}

\begin{document}

% generate title
\maketitle

\noindent
The aim of the \texttt{adharticle} \LaTeX{} class is to provide a simple and
clean way to set up the document as I like it. This could easily be achieved
through the \LaTeX{} document header, but giving by combining these settings
into a class:
\begin{enumerate}
\item The need to duplicate options and settings in each document is avoided,
\item If the settings are changed, they need only be modified in one location
  and all documents using this class may be recompiled without modification.
\end{enumerate}

This class will be used primarily for long form writing, producing A4 documents
with a layout aimed at reading longer text. It will give a good typesetting for
reading and will include all necessary packages for typesetting maths, tables,
Greek/Hebrew text and source code.

\section{Compilation}
\label{sec:compilation}

The \texttt{adharticle} class is designed for use with \LaTeX{} and \XeLaTeX{},
usually producing PDF files. The option \texttt{--shell-escape} must be provided
to the compiler as the \texttt{minted} package calls external commands.


\section{Class Options}

The following options are available to the \texttt{adharticle} document class:
\begin{description}
  % [todo] - make footnote link a url once hyperref package implemented.
\item[fullpage] If \texttt{fullpage} option is present, use the fullpage
  package\footnote{http://www.ctan.org/pkg/fullpage}. This reduces the margin
  sizes on all sizes to $1\,\mathrm{in}$.
\item[paper/electronic] If \texttt{paper} is set, the document is intended for
  printing on paper. If \texttt{electronic} is set, the document is intended for
  viewing electronically (probably PDF). The default settings reflect the medium
  of intended use. A script should be written to produce both if necessary to
  avoid 2 versions of the files floating around. Default to electronic
  behaviour.
\end{description}

% [todo] - write a script to produce both paper and electronic copies if
% necessary

\section{Included Packages and Settings}

% [todo] - add subsection on dtklogos

This section details the packages used in the \texttt{adharticle} class and
their settings.

\subsection{inputenc}

The \texttt{inputenc} package\footnote{http://ctan.org/pkg/inputenc} allows use
of different file encodings for the LaTeX file. This is set by:

\latex|\RequirePackage[utf8]{inputenc}|
Using the \texttt{utf8} encoding allows Unicode characters to be easily
inputted, especially useful for including Greek text or other non-western
characters.

\subsubsection{A sub-sub-section}

Just testing to see what it looks like.

\paragraph{A paragraph}

Again, just testing what it looks like.

\subparagraph{A sub-paragraph}

Testing what this looks like too.

This is some text in the normal typeface.

\textbf{This is some text in the bold typeface.}

\textsf{This is some sans-serif type.}

\texttt{This is some monospaced type.}

\textsc{Some text with Small Caps.}

\selectlanguage{greek}

ὁ θεος λεγει τον δουλον.

\selectlanguage{english}

\subsection{babel}

The \texttt{babel} package\footnote{http://ctan.org/pkg/babel} provides support
for language specific hyphenation patterns and also culturally and
linguistically determined typographical rules. Babel is configured by:

\latex|\RequirePackage[greek,english]{babel}|
Babel is configured to provide support for Greek and for English, with English
as the default. Use:

\latex|\selectlanguage{Greek}|
to change to Greek (or other languages as appropriate), and similarly back to
English.

\subsection{color}

The \texttt{color} package\footnote{http://ctan.org/pkg/color} provides colour
management. Standard colour set is provided, new colours may be defined using:

\latex|\definecolor{name}{model (e.g. rgb)}{colour specification}|
Colours may then be applied to text or other elements. Especially useful when
drawing pictures with Tikz/pgf.

\subsection{graphicx}

The \texttt{graphicx} package\footnote{http://ctan.org/pkg/graphicx} allows
inclusion of graphics in the document. A graphic may be included by giving the
command:

\subsection{placeins}

The \texttt{placeins} package defines the command
\texttt{\textbackslash{}FloatBarrier} which stops floats from moving past.

\subsection{amssymb, amsmath}

The packages \texttt{amssymb} and \texttt{amsmath} provide support for
additional symbols and mathematical notation.


\subsection{ctable}

\texttt{ctable} is a powerful table package which uses \texttt{booktabs} to make
a nice looking table and also provides footnotes for the table. Default values
for options of \texttt{ctable} are set in the class file.

\subsection{minted}

The \texttt{minted} package provides syntax highlighting for source code
included in the document, along with a number of other useful features for
including code.

\subsection{fancyvrb}

The \texttt{fancyvrb} package provides a more powerful version of the
\texttt{verbatim} environment.

\subsection{microtype}

The \texttt{microtype} package provides access to micro-typographical extensions
to improve document layout.

\subsection{hyperref}

Makes references in the document into clickable hyperlinks for easy
navigation. Only used for documents intended for electronic files (not for paper
documents). For example, a reference to Section~\ref{sec:compilation}. Internet
links can also be used: \href{http://www.google.co.uk}{google}.

\end{document}

%%% Local Variables:
%%% mode: latex
%%% TeX-master: t
%%% End:
