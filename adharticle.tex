% File name: /home/adh/Projects/LaTeXClasses/adharticle_test.tex
% Document to test functionality of adharticle class.
% Author: adh
% Date: Wed 24 Dec 2014 17:23

\documentclass{adharticle}  % Standard document class

% Define title here:
\title{A customised \LaTeX{} article class}
\author{Andy Holt}
\date{\today}

\begin{document}

% generate title
\maketitle

\noindent
The aim of the \texttt{adharticle} \LaTeX{} class is to provide a
simple and clean way to set up the document as I like it. This could easily be
achieved through the \LaTeX{} document header, but giving by combining these
settings into a class:
\begin{enumerate}
\item The need to duplicate options and settings in each document is avoided,
\item If the settings are changed, they need only be modified in one location
  and all documents using this class may be recompiled without modification.
\end{enumerate}

This class will be used primarily for longform writing, producing A4 documents
with a layout aimed at reading longer text. It will give a good typesetting for
reading and will include all necessary packages for typesetting maths, tables,
Greek/Hebrew text and source code.

\section{Class Options}

The following options are available to the \texttt{adharticle} document class:
\begin{description}
\item[fullpage] If \textt{fullpage} option is present, use the fullpage
  package. This reduces the margin sizes on all sizes to $1\,\mathrm{in}$.
\item[Item 2] If \texttt{paper} is set, the document is intended for printing on
  paper. If \texttt{electronic} is set, the document is intended for viewing
  electronically (probably PDF). The default settings reflect the medium of
  intended use. A script should be written to produce both if necessary to
  avoide 2 versions of the files floating around. Default to electronic behaviour.
\end{description}

% [todo] - write a script to produce both paper and electronic copies if necessary

\section{Included Packages and Settings}

This section details the packages used in the \texttt{adharticle} class and
their settings.

\end{document}
