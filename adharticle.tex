% File name: /home/adh/Projects/LaTeXClasses/adharticle_test.tex
% Document to test functionality of adharticle class.
% Author: adh
% Date: Wed 24 Dec 2014 17:23

\documentclass{adharticle}  % Standard document class

% Define title here:
\title{A customised \LaTeX{} article class}
\author{Andy Holt}
\date{\today}

\begin{document}

% generate title
\maketitle

\noindent
The aim of the \texttt{adharticle} \LaTeX{} class is to provide a
simple and clean way to set up the document as I like it. This could easily be
achieved through the \LaTeX{} document header, but giving by combining these
settings into a class:
\begin{enumerate}
\item The need to duplicate options and settings in each document is avoided,
\item If the settings are changed, they need only be modified in one location
  and all documents using this class may be recompiled without modification.
\end{enumerate}

This class will be used primarily for longform writing, producing A4 documents
with a layout aimed at reading longer text. It will give a good typesetting for
reading and will include all necessary packages for typesetting maths, tables,
Greek/Hebrew text and source code.

\section{Class Options}

The following options are available to the \texttt{adharticle} document class:
\begin{description}
% [todo] - make footnote link a url once hyperref package implemented.
\item[fullpage] If \texttt{fullpage} option is present, use the fullpage
  package\footnote{http://www.ctan.org/pkg/fullpage}. This reduces the margin sizes on all sizes to $1\,\mathrm{in}$.
\item[paper/electronic] If \texttt{paper} is set, the document is intended for
  printing on paper. If \texttt{electronic} is set, the document is intended for
  viewing electronically (probably PDF). The default settings reflect the medium
  of intended use. A script should be written to produce both if necessary to
  avoide 2 versions of the files floating around. Default to electronic
  behaviour.
\end{description}

% [todo] - write a script to produce both paper and electronic copies if necessary

\section{Included Packages and Settings}

This section details the packages used in the \texttt{adharticle} class and
their settings.

\subsection{inputenc}

The \texttt{inputenc} package\footnote{http://ctan.org/pkg/inputenc} allows use
of different file encodings for the LaTeX file. This is set by:
% [todo] - add input encoding setting line from .cls file once minted implemented
Using the \texttt{utf8} encoding allows unicode characters to be easily
inputted, especially useful for including Greek text or other non-western
characters.

\subsection{babel}

The \texttt{babel} package\footnote{http://ctan.org/pkg/babel} provides support
for language specific hyphenation patterns and also culturally and
linguistically determined typographical rules. Babel is configured by:
% [todo] - add babel setting line from .cls file once minted implemented
Babel is configured to provide support for Greek and for English, with English
as the default. Use:
% [todo] - add ``\selectlanguage{Greek}'' once minted implemented
to change to Greek (or other  languages as appropriate), and similarly back to
English.

\subsection{color}

The \texttt{color} package\footnote{http://ctan.org/pkg/color} provides colour
management. Standard colour set is provided, new colours may be defined using:
% [todo] - include code to define a colour.
Colours may then be applied to text or other elements. Especially useful when
drawing pictures with Tikz/pgf.

\subsection{graphicx}

The \textt{graphicx} package\footnote{http://ctan.org/pkg/graphicx} allows
inclusion of graphics in the document. A graphic may be included by giving the
command:

\subsection{placeins}

The \texttt{placeins} package defines the command \texttt{\textbackslash
  FloatBarrier} which stops floats from moving past.

\subsection{amssymb, amsmath}

The packages \texttt{amssymb} and \texttt{amsmath} provide support for additional
symbols and mathematical notation.


\subsection{ctable}

\texttt{ctable} is a powerful table package which uses \texttt{booktabs} to make
a nice looking table and also provides footnotes for the table. Default values
for options of \texttt{ctable} are set in the class file.

\end{document}
